% Author Rigoberto Leander Salgado Reyes <rlsalgado2006@gmail.com>
% 
% Copyright 2016 by Rigoberto Leander Salgado Reyes.
% 
% This program is licensed to you under the terms of version 3 of the
% GNU Affero General Public License. This program is distributed WITHOUT
% ANY EXPRESS OR IMPLIED WARRANTY, INCLUDING THOSE OF NON-INFRINGEMENT,
% MERCHANTABILITY OR FITNESS FOR A PARTICULAR PURPOSE. Please refer to the
% AGPL (http:www.gnu.org/licenses/agpl-3.0.txt) for more details.


% Ajustar los margenes de la hoja (el izquierdo es mayor que el derecho para el encuadernado).
% tmargin => superior
% rmargin => derecho
% bmargin => inferior
% lmargin => izquierdo.
\usepackage[tmargin=3.81cm, rmargin=1.2cm, bmargin=3.31cm, lmargin=2cm]{geometry}

% Utilizar la codificación UTF8 para el documento, de esta forma se puede utilizar las tildes
% y la eñe de la forma: áéíóúñ y no \'a \'e \'i \'o \'u \~n. Se escribe mejor y ayuda en la 
% corrección ortográfica.
\usepackage[utf8x]{inputenc}
\usepackage{ucs}

% Utilizar el idioma español.
\usepackage[spanish]{babel}

% Se utiliza para crear comandos de una forma más amigable.
\usepackage{xparse}

% Se utiliza entre otras cosas para cargar imágenes.
\usepackage{graphicx}

% El paquete acronym brinda la posibilidad de utilizar los acronimos de forma sencilla.
\usepackage{acronym}

% Se utiliza para mostrar algoritmos.
\usepackage{algorithm}

% Usar interlineado de 1.5
% para que coincida con el interlineado 1.5 del ms-word el valor debe ser de 1.25.
% existen varias formas de cambiar el interlineado.
\linespread{1.5}

% El paquete fancyhdr es usado para las cabeceras y pies de página.
\usepackage{fancyhdr}

\fancypagestyle{plain}{
  \fancyhf{} % Restablece la configuración por defecto de las cabecera y pies de página.
  \fancyhead{} % Restablece la configuración por defecto de las cabecera.
  \fancyfoot{} % Restablece la configuración por defecto de los pies de página.
  \fancyfoot[R]{\thepage}
%   \fancyfoot[RO]{\thepage}
%   \fancyfoot[LE]{\thepage}
  % El estilo fancy agrega una línea que delimita el encabezado y pie de página.
  % cambiando su grosor a 0 evitamos que aparezca.
  \renewcommand{\headrulewidth}{0pt}
  \renewcommand{\footrulewidth}{0pt}
}

\pagestyle{fancy}

% Fuente clon de Arial
\renewcommand{\rmdefault}{phv}
\renewcommand{\sfdefault}{phv}

